\section{Introduction and usage}% ===> this file was generated automatically by noweave --- better not edit it

The {\Tt{}noweb.mk\nwendquote} include provides suffix rules for weaving and tangling 
(produce documentation and code, respectively).
The framework uses the conventions of make~\cite[Section~10.2]{GNUMake}.

The suffix rules of make works by taking a prerequisite with one suffix and 
applying the recipe to get a target with another suffix.
This requires the stem of the filename to be identical.
The framework can handle files of suffixes defined in {\Tt{}NOWEB{\_}SUFFIXES\nwendquote}.
However, we also provide {\Tt{}NOWEAVE.tex\nwendquote} to weave a TeX file ({\Tt{}.tex\nwendquote}) from a 
NOWEB file ({\Tt{}.nw\nwendquote}) and {\Tt{}NOTANGLE.suf\nwendquote} to tangle a {\Tt{}.suf\nwendquote} file from a 
NOWEB file ({\Tt{}.nw\nwendquote}).
There is also a {\Tt{}NOWEAVE.pdf\nwendquote} to weave directly to PDF.
This assumes that the file is independent, \ie no special LaTeX preamble.


\section{Implementation}

The overall structure is the same as for other include files.
We will cover the suffix rules for documentation first and then those for code.
\nwfilename{noweb.mk.nw}\nwbegincode{1}\sublabel{NW3JNJ0B-7y8Yg-1}\nwmargintag{{\nwtagstyle{}\subpageref{NW3JNJ0B-7y8Yg-1}}}\moddef{\code{}noweb.mk\edoc{}~{\nwtagstyle{}\subpageref{NW3JNJ0B-7y8Yg-1}}}\endmoddef\nwstartdeflinemarkup\nwenddeflinemarkup
ifndef NOWEB_MK
NOWEB_MK = true

\LA{}variables~{\nwtagstyle{}\subpageref{NW3JNJ0B-29jjeg-1}}\RA{}

INCLUDE_MAKEFILES?=.
MAKEFILES_DIR?=$\{INCLUDE_MAKEFILES\}
include $\{MAKEFILES_DIR\}/tex.mk

\LA{}suffix rules for weaving documentation~{\nwtagstyle{}\subpageref{NW3JNJ0B-dAOzE-1}}\RA{}
\LA{}suffix rules for tangling code~{\nwtagstyle{}\subpageref{NW3JNJ0B-HHfVd-1}}\RA{}

endif
\nwnotused{[[noweb.mk]]}\nwendcode{}\nwbegindocs{2}\nwdocspar

\subsection{Weaving documentation}

We will use the {\Tt{}noweave\nwendquote} command to weave the documentation.
We are interested in two cases:
\begin{enumerate}
\item when a source program should be converted to TeX to be included in a 
larger document, and
\item when a source program is independent and should be converted to PDF.
\end{enumerate}

The order of the rules are important.
To ensure make takes the \enquote{shortcut} of the second case, we must specify 
that rule first.
\nwenddocs{}\nwbegincode{3}\sublabel{NW3JNJ0B-dAOzE-1}\nwmargintag{{\nwtagstyle{}\subpageref{NW3JNJ0B-dAOzE-1}}}\moddef{suffix rules for weaving documentation~{\nwtagstyle{}\subpageref{NW3JNJ0B-dAOzE-1}}}\endmoddef\nwstartdeflinemarkup\nwusesondefline{\\{NW3JNJ0B-7y8Yg-1}}\nwenddeflinemarkup
\LA{}suffix rule for weaving to PDF~{\nwtagstyle{}\subpageref{NW3JNJ0B-4Di5Ld-1}}\RA{}
\LA{}suffix rule for weaving to TeX~{\nwtagstyle{}\subpageref{NW3JNJ0B-3EfcsT-1}}\RA{}
\nwused{\\{NW3JNJ0B-7y8Yg-1}}\nwendcode{}\nwbegindocs{4}\nwdocspar

Now, for the first case, we let
\nwenddocs{}\nwbegincode{5}\sublabel{NW3JNJ0B-29jjeg-1}\nwmargintag{{\nwtagstyle{}\subpageref{NW3JNJ0B-29jjeg-1}}}\moddef{variables~{\nwtagstyle{}\subpageref{NW3JNJ0B-29jjeg-1}}}\endmoddef\nwstartdeflinemarkup\nwusesondefline{\\{NW3JNJ0B-7y8Yg-1}}\nwprevnextdefs{\relax}{NW3JNJ0B-29jjeg-2}\nwenddeflinemarkup
NOWEAVE.tex?=       noweave $\{NOWEAVEFLAGS.tex\} $< > $@
NOWEAVEFLAGS.tex?=  $\{NOWEAVEFLAGS\} -x -n -delay -t2
\nwalsodefined{\\{NW3JNJ0B-29jjeg-2}\\{NW3JNJ0B-29jjeg-3}\\{NW3JNJ0B-29jjeg-4}\\{NW3JNJ0B-29jjeg-5}\\{NW3JNJ0B-29jjeg-6}}\nwused{\\{NW3JNJ0B-7y8Yg-1}}\nwendcode{}\nwbegindocs{6}Now we need to specify all the suffixes to use and then construct suffix rules 
for all of them.
Fortunately we can use the same recipe for all, so we only need to write one 
recipe for multiple targets.
We will use a variable {\Tt{}NOWEB{\_}SUFFIXES\nwendquote} to keep a list of supported suffixes.  
Since these suffixes only matter for tangling, we will set the variable in that
section.
For now, we only use it.
\nwenddocs{}\nwbegincode{7}\sublabel{NW3JNJ0B-3EfcsT-1}\nwmargintag{{\nwtagstyle{}\subpageref{NW3JNJ0B-3EfcsT-1}}}\moddef{suffix rule for weaving to TeX~{\nwtagstyle{}\subpageref{NW3JNJ0B-3EfcsT-1}}}\endmoddef\nwstartdeflinemarkup\nwusesondefline{\\{NW3JNJ0B-dAOzE-1}}\nwenddeflinemarkup
%.tex: %.nw
  $\{NOWEAVE.tex\}

define def_weave_to_tex
%.tex: %$(1).nw
  $$\{NOWEAVE.tex\}
endef

$(foreach suf,$\{NOWEB_SUFFIXES\},$(eval $(call def_weave_to_tex,$\{suf\})))
\nwused{\\{NW3JNJ0B-dAOzE-1}}\nwendcode{}\nwbegindocs{8}\nwdocspar

To differentiate the second case from the first (in terms of suffix rules), we 
go from {\Tt{}.nw\nwendquote} directly to {\Tt{}.pdf\nwendquote}\footnote{%
  Note, however, that these pattern rules will never be used by make.
  The make algorithm performs a depth-first search, thus make will take the 
  long way by first converting to TeX, then to PDF.
  We can determine which of these two pattern rules should be used by moving 
  the inclusion of the {\Tt{}tex.mk\nwendquote} include file above.
}.
\nwenddocs{}\nwbegincode{9}\sublabel{NW3JNJ0B-4Di5Ld-1}\nwmargintag{{\nwtagstyle{}\subpageref{NW3JNJ0B-4Di5Ld-1}}}\moddef{suffix rule for weaving to PDF~{\nwtagstyle{}\subpageref{NW3JNJ0B-4Di5Ld-1}}}\endmoddef\nwstartdeflinemarkup\nwusesondefline{\\{NW3JNJ0B-dAOzE-1}}\nwenddeflinemarkup
%.pdf: %.nw
  $\{NOWEAVE.pdf\}

define def_weave_to_pdf
%.pdf: %$(1).nw
  $$\{NOWEAVE.pdf\}
endef

$(foreach suf,$\{NOWEB_SUFFIXES\},$(eval $(call def_weave_to_pdf,$\{suf\})))
\nwused{\\{NW3JNJ0B-dAOzE-1}}\nwendcode{}\nwbegindocs{10}What differs {\Tt{}NOWEAVE.pdf\nwendquote} from {\Tt{}NOWEAVE.tex\nwendquote} is the options to 
{\Tt{}noweave\nwendquote} and the compilation step (instead of having that separately).
\nwenddocs{}\nwbegincode{11}\sublabel{NW3JNJ0B-29jjeg-2}\nwmargintag{{\nwtagstyle{}\subpageref{NW3JNJ0B-29jjeg-2}}}\moddef{variables~{\nwtagstyle{}\subpageref{NW3JNJ0B-29jjeg-1}}}\plusendmoddef\nwstartdeflinemarkup\nwusesondefline{\\{NW3JNJ0B-7y8Yg-1}}\nwprevnextdefs{NW3JNJ0B-29jjeg-1}{NW3JNJ0B-29jjeg-3}\nwenddeflinemarkup
NOWEAVE.pdf?=       \\
  noweave $\{NOWEAVEFLAGS.pdf\} $< > $\{@:.pdf=.tex\} && \\
  latexmk -pdf $\{@:.pdf=.tex\}
NOWEAVEFLAGS.pdf?=  \\
  $\{NOWEAVEFLAGS\} -x -t2 \\
  -option "shift,breakcode,longxref,longchunks"
\nwused{\\{NW3JNJ0B-7y8Yg-1}}\nwendcode{}\nwbegindocs{12}\nwdocspar

\subsection{Tangling code}

We will now cover the rules for tangling the source code for different 
languages.
\nwenddocs{}\nwbegincode{13}\sublabel{NW3JNJ0B-HHfVd-1}\nwmargintag{{\nwtagstyle{}\subpageref{NW3JNJ0B-HHfVd-1}}}\moddef{suffix rules for tangling code~{\nwtagstyle{}\subpageref{NW3JNJ0B-HHfVd-1}}}\endmoddef\nwstartdeflinemarkup\nwusesondefline{\\{NW3JNJ0B-7y8Yg-1}}\nwenddeflinemarkup
\LA{}general tangling rules~{\nwtagstyle{}\subpageref{NW3JNJ0B-3Kk0zj-1}}\RA{}
\LA{}special rules for different languages~{\nwtagstyle{}\subpageref{NW3JNJ0B-3iZmQi-1}}\RA{}
\nwused{\\{NW3JNJ0B-7y8Yg-1}}\nwendcode{}\nwbegindocs{14}We will first write some general pattern rules, then supply ways to adapt 
this rule to the different languages.

\subsubsection{Handling underscores in filenames}

A critical design decision concerns how we reference chunk names in the 
Makefile rules.
Many programming languages (particularly Python) use underscores in filenames, 
such as {\Tt{}module{\_}name.py\nwendquote} or {\Tt{}attachment{\_}cache.py\nwendquote}.

When these filenames appear in chunk definitions like {\Tt{}\LA{}module\_name.py~{\nwtagstyle{}\subpageref{nw@notdef}}\RA{}=\nwendquote}, 
LaTeX interprets the underscores as subscript commands during documentation 
generation (weaving), causing compilation errors.
The error manifests as \enquote{Missing \$ inserted} because LaTeX expects 
math mode for subscripts.

\paragraph{The {\Tt{}[[...]]\nwendquote} notation solution}

Noweb provides the {\Tt{}[[...]]\nwendquote} notation specifically to handle special 
characters in code references.
When we write {\Tt{}\LA{}\code{}module{\_}name.py\edoc{}~{\nwtagstyle{}\subpageref{nw@notdef}}\RA{}=\nwendquote} in a {\Tt{}.nw\nwendquote} file, noweb 
automatically escapes all LaTeX special characters (\_, \&, \%, etc.) when 
weaving documentation.
The brackets tell noweb: \enquote{treat this as code, not LaTeX}.

\paragraph{Why we use it in Makefile rules}

Since our Makefile rules must match the chunk names used in {\Tt{}.nw\nwendquote} files, and 
we want all Python files to use {\Tt{}[[...]]\nwendquote} notation (to handle underscores), 
we must specify {\Tt{}-R"[[filename\nwendquote}"]] in the notangle command.
The double quotes protect the brackets from shell interpretation, and notangle 
then looks for a chunk named {\Tt{}[[filename]]\nwendquote}.

This standardization means:
\begin{enumerate}
\item All Python chunk definitions use {\Tt{}\LA{}\code{}filename.py\edoc{}~{\nwtagstyle{}\subpageref{nw@notdef}}\RA{}=\nwendquote}
\item All Makefile rules use {\Tt{}-R"[[{\$}(notdir\ {\$}@)\nwendquote}"]]
\item Underscores work without escaping
\item Consistent pattern across the project
\end{enumerate}

\paragraph{Alternative approaches rejected}

We considered but rejected:
\begin{description}
\item[Escaping underscores] Writing {\Tt{}{\nwbackslash}{\nwbackslash}{\_}\nwendquote} in chunk names requires escaping 
everywhere the chunk is referenced, making the {\Tt{}.nw\nwendquote} file harder to read.
\item[Using hyphens] Chunk names like {\Tt{}module-name.py\nwendquote} avoid LaTeX issues but 
require Makefile renaming rules ({\Tt{}-name.py\nwendquote} to {\Tt{}{\_}name.py\nwendquote}), adding 
complexity.
\item[No special characters] Restricting filenames to avoid underscores breaks 
Python conventions where {\Tt{}module{\_}name\nwendquote} is idiomatic.
\end{description}

The {\Tt{}[[...]]\nwendquote} notation approach handles all special characters uniformly and 
keeps {\Tt{}.nw\nwendquote} files readable.

We will use notangle(1).
Note that we use the noweb {\Tt{}[[...]]\nwendquote} notation to quote the chunk name.
This is critical for handling filenames with underscores (common in Python) or 
other LaTeX special characters.
Without the brackets, chunk names like {\Tt{}\LA{}module\_name.py~{\nwtagstyle{}\subpageref{nw@notdef}}\RA{}\nwendquote} would cause 
LaTeX to interpret the underscore as a subscript command, breaking 
documentation generation.
The {\Tt{}[[...]]\nwendquote} notation tells noweb to escape all special characters properly.

\nwenddocs{}\nwbegincode{15}\sublabel{NW3JNJ0B-29jjeg-3}\nwmargintag{{\nwtagstyle{}\subpageref{NW3JNJ0B-29jjeg-3}}}\moddef{variables~{\nwtagstyle{}\subpageref{NW3JNJ0B-29jjeg-1}}}\plusendmoddef\nwstartdeflinemarkup\nwusesondefline{\\{NW3JNJ0B-7y8Yg-1}}\nwprevnextdefs{NW3JNJ0B-29jjeg-2}{NW3JNJ0B-29jjeg-4}\nwenddeflinemarkup
NOTANGLEFLAGS?=
NOTANGLE?=      notangle $\{NOTANGLEFLAGS\} -R"[[$(notdir $@)]]" $(filter %.nw,$^) | \\
                  $\{CPIF\} $@ && noroots $(filter %.nw,$^)
\nwused{\\{NW3JNJ0B-7y8Yg-1}}\nwendcode{}\nwbegindocs{16}We will also use the command cpif(1).
This command only updates the files if they have changed.
We need this since many files may reside in the same NOWEB source file, but 
only some of them are updated.
Without {\Tt{}cpif\nwendquote}, make would normally \emph{update all files} if \emph{any has 
changed} --- which is clearly undesirable.
\nwenddocs{}\nwbegincode{17}\sublabel{NW3JNJ0B-29jjeg-4}\nwmargintag{{\nwtagstyle{}\subpageref{NW3JNJ0B-29jjeg-4}}}\moddef{variables~{\nwtagstyle{}\subpageref{NW3JNJ0B-29jjeg-1}}}\plusendmoddef\nwstartdeflinemarkup\nwusesondefline{\\{NW3JNJ0B-7y8Yg-1}}\nwprevnextdefs{NW3JNJ0B-29jjeg-3}{NW3JNJ0B-29jjeg-5}\nwenddeflinemarkup
CPIF?=          cpif
\nwused{\\{NW3JNJ0B-7y8Yg-1}}\nwendcode{}\nwbegindocs{18}However, since we use this variable, cpif(1) can be substituted for tee(1) in 
desirable situations.

\paragraph{General pattern rules}

There are two general pattern rules that we will add.
\nwenddocs{}\nwbegincode{19}\sublabel{NW3JNJ0B-3Kk0zj-1}\nwmargintag{{\nwtagstyle{}\subpageref{NW3JNJ0B-3Kk0zj-1}}}\moddef{general tangling rules~{\nwtagstyle{}\subpageref{NW3JNJ0B-3Kk0zj-1}}}\endmoddef\nwstartdeflinemarkup\nwusesondefline{\\{NW3JNJ0B-HHfVd-1}}\nwenddeflinemarkup
\LA{}tangle source files with suffix~{\nwtagstyle{}\subpageref{NW3JNJ0B-1VVpCX-1}}\RA{}
\LA{}tangle source files without suffix~{\nwtagstyle{}\subpageref{NW3JNJ0B-2tqDpo-1}}\RA{}
\nwused{\\{NW3JNJ0B-HHfVd-1}}\nwendcode{}\nwbegindocs{20}In the first one, we will tangle a file with suffix {\Tt{}.suf\nwendquote} from the source 
file with suffix {\Tt{}.suf.nw\nwendquote} and in the second a source file with suffix 
{\Tt{}.nw\nwendquote}.

We can start with the second.
In this rule, we have a file with a supported suffix {\Tt{}.suf\nwendquote} depend on the 
NOWEB source file with suffix {\Tt{}.nw\nwendquote}.
Then we let the recipe be set by the variable {\Tt{}NOTANGLE.suf\nwendquote}, which is the 
convention followed by make(1)~\cite[Sect.\ 10.2]{GNUMake}.
\nwenddocs{}\nwbegincode{21}\sublabel{NW3JNJ0B-2tqDpo-1}\nwmargintag{{\nwtagstyle{}\subpageref{NW3JNJ0B-2tqDpo-1}}}\moddef{tangle source files without suffix~{\nwtagstyle{}\subpageref{NW3JNJ0B-2tqDpo-1}}}\endmoddef\nwstartdeflinemarkup\nwusesondefline{\\{NW3JNJ0B-3Kk0zj-1}}\nwenddeflinemarkup
$(addprefix %,$\{NOWEB_SUFFIXES\}): %.nw
  $\{NOTANGLE$(suffix $@)\}
\nwused{\\{NW3JNJ0B-3Kk0zj-1}}\nwendcode{}\nwbegindocs{22}\nwdocspar

For the other case, we add rules using the suffix prefixed to {\Tt{}.nw\nwendquote}.
\nwenddocs{}\nwbegincode{23}\sublabel{NW3JNJ0B-1VVpCX-1}\nwmargintag{{\nwtagstyle{}\subpageref{NW3JNJ0B-1VVpCX-1}}}\moddef{tangle source files with suffix~{\nwtagstyle{}\subpageref{NW3JNJ0B-1VVpCX-1}}}\endmoddef\nwstartdeflinemarkup\nwusesondefline{\\{NW3JNJ0B-3Kk0zj-1}}\nwenddeflinemarkup
define with_suffix_target
%$(1): %$(1).nw
  $$\{NOTANGLE$$(suffix $$@)\}
endef
$(foreach suf,$\{NOWEB_SUFFIXES\},$(eval $(call with_suffix_target,$\{suf\})))
\nwused{\\{NW3JNJ0B-3Kk0zj-1}}\nwendcode{}\nwbegindocs{24}However, this rule does not capture some of the things we want, \eg we cannot 
tangle a header file {\Tt{}.h\nwendquote} from a {\Tt{}.cpp.nw\nwendquote} file.
We must add these rules manually, which we do below.

\paragraph{Rules for different languages}

We will now cover specialized instances of the general pattern rules defined 
above.
We will simply set the default variables.
\nwenddocs{}\nwbegincode{25}\sublabel{NW3JNJ0B-29jjeg-5}\nwmargintag{{\nwtagstyle{}\subpageref{NW3JNJ0B-29jjeg-5}}}\moddef{variables~{\nwtagstyle{}\subpageref{NW3JNJ0B-29jjeg-1}}}\plusendmoddef\nwstartdeflinemarkup\nwusesondefline{\\{NW3JNJ0B-7y8Yg-1}}\nwprevnextdefs{NW3JNJ0B-29jjeg-4}{NW3JNJ0B-29jjeg-6}\nwenddeflinemarkup
\LA{}defaults for C and C++~{\nwtagstyle{}\subpageref{NW3JNJ0B-1RkDZv-1}}\RA{}
\LA{}defaults for Haskell~{\nwtagstyle{}\subpageref{NW3JNJ0B-3G3VYa-1}}\RA{}
\LA{}defaults for Make~{\nwtagstyle{}\subpageref{NW3JNJ0B-zwVnh-1}}\RA{}
\LA{}defaults for remaining~{\nwtagstyle{}\subpageref{NW3JNJ0B-34DmtX-1}}\RA{}
\nwused{\\{NW3JNJ0B-7y8Yg-1}}\nwendcode{}\nwbegindocs{26}As noted above, we need some special rules for the C and C++ header files, 
but no extra rules for any other language.
\nwenddocs{}\nwbegincode{27}\sublabel{NW3JNJ0B-3iZmQi-1}\nwmargintag{{\nwtagstyle{}\subpageref{NW3JNJ0B-3iZmQi-1}}}\moddef{special rules for different languages~{\nwtagstyle{}\subpageref{NW3JNJ0B-3iZmQi-1}}}\endmoddef\nwstartdeflinemarkup\nwusesondefline{\\{NW3JNJ0B-HHfVd-1}}\nwenddeflinemarkup
\LA{}rules for C and C++~{\nwtagstyle{}\subpageref{NW3JNJ0B-1mCQja-1}}\RA{}
\nwused{\\{NW3JNJ0B-HHfVd-1}}\nwendcode{}\nwbegindocs{28}\nwdocspar

For the languages of the C-family, we will use the {\Tt{}-L\nwendquote} option to get the 
line preprocessor-directive in the generated source --- this will allow {\Tt{}gdb\nwendquote} 
and the compiler to point to lines in the NOWEB source file, and not to the 
generated file.
\nwenddocs{}\nwbegincode{29}\sublabel{NW3JNJ0B-1RkDZv-1}\nwmargintag{{\nwtagstyle{}\subpageref{NW3JNJ0B-1RkDZv-1}}}\moddef{defaults for C and C++~{\nwtagstyle{}\subpageref{NW3JNJ0B-1RkDZv-1}}}\endmoddef\nwstartdeflinemarkup\nwusesondefline{\\{NW3JNJ0B-29jjeg-5}}\nwprevnextdefs{\relax}{NW3JNJ0B-1RkDZv-2}\nwenddeflinemarkup
NOWEB_SUFFIXES+=    .c .cc .cpp .cxx
NOTANGLEFLAGS.c?=   -L
NOTANGLE.c?=        notangle $\{NOTANGLEFLAGS.c\} -R"[[$(notdir $@)]]" \\
  $(filter %.nw,$^) | $\{CPIF\} $@ && noroots $(filter %.nw,$^)
NOTANGLEFLAGS.cc?=  $\{NOTANGLEFLAGS.c\}
NOTANGLE.cc?=       notangle $\{NOTANGLEFLAGS.cc\} -R"[[$(notdir $@)]]" \\
  $(filter %.nw,$^) | $\{CPIF\} $@ && noroots $(filter %.nw,$^)
NOTANGLEFLAGS.cpp?= $\{NOTANGLEFLAGS.c\}
NOTANGLE.cpp?=      notangle $\{NOTANGLEFLAGS.cpp\} -R"[[$(notdir $@)]]" \\
  $(filter %.nw,$^) | $\{CPIF\} $@ && noroots $(filter %.nw,$^)
NOTANGLEFLAGS.cxx?= $\{NOTANGLEFLAGS.c\}
NOTANGLE.cxx?=      notangle $\{NOTANGLEFLAGS.cxx\} -R"[[$(notdir $@)]]" \\
  $(filter %.nw,$^) | $\{CPIF\} $@ && noroots $(filter %.nw,$^)
\nwalsodefined{\\{NW3JNJ0B-1RkDZv-2}}\nwused{\\{NW3JNJ0B-29jjeg-5}}\nwendcode{}\nwbegindocs{30}\nwdocspar

For C-family source code, we will assume that the header files (declarations) 
are written together with the definitions, so that we can extract both files 
from the same NOWEB source.
However, for this we must add extra pattern rules.
\nwenddocs{}\nwbegincode{31}\sublabel{NW3JNJ0B-1mCQja-1}\nwmargintag{{\nwtagstyle{}\subpageref{NW3JNJ0B-1mCQja-1}}}\moddef{rules for C and C++~{\nwtagstyle{}\subpageref{NW3JNJ0B-1mCQja-1}}}\endmoddef\nwstartdeflinemarkup\nwusesondefline{\\{NW3JNJ0B-3iZmQi-1}}\nwenddeflinemarkup
%.h: %.c.nw
  $\{NOTANGLE.h\}

%.hh: %.cc.nw
  $\{NOTANGLE.hh\}

%.hpp: %.cpp.nw
  $\{NOTANGLE.hpp\}

%.hxx: %.cxx.nw
  $\{NOTANGLE.hxx\}
\nwused{\\{NW3JNJ0B-3iZmQi-1}}\nwendcode{}\nwbegindocs{32}Finally, we can define the variables used for tangling.
\nwenddocs{}\nwbegincode{33}\sublabel{NW3JNJ0B-1RkDZv-2}\nwmargintag{{\nwtagstyle{}\subpageref{NW3JNJ0B-1RkDZv-2}}}\moddef{defaults for C and C++~{\nwtagstyle{}\subpageref{NW3JNJ0B-1RkDZv-1}}}\plusendmoddef\nwstartdeflinemarkup\nwusesondefline{\\{NW3JNJ0B-29jjeg-5}}\nwprevnextdefs{NW3JNJ0B-1RkDZv-1}{\relax}\nwenddeflinemarkup
NOWEB_SUFFIXES+=    .h .hh .hpp .hxx
NOTANGLEFLAGS.h?=   -L
NOTANGLE.h?=        notangle $\{NOTANGLEFLAGS.h\} -R"[[$(notdir $@)]]" \\
  $(filter %.nw,$^) | $\{CPIF\} $@ && noroots $(filter %.nw,$^)
NOTANGLEFLAGS.hh?=  $\{NOTANGLEFLAGS.h\}
NOTANGLE.hh?=       notangle $\{NOTANGLEFLAGS.hh\} -R"[[$(notdir $@)]]" \\
  $(filter %.nw,$^) | $\{CPIF\} $@ && noroots $(filter %.nw,$^)
NOTANGLEFLAGS.hpp?= $\{NOTANGLEFLAGS.h\}
NOTANGLE.hpp?=      notangle $\{NOTANGLEFLAGS.hpp\} -R"[[$(notdir $@)]]" \\
  $(filter %.nw,$^) | $\{CPIF\} $@ && noroots $(filter %.nw,$^)
NOTANGLEFLAGS.hxx?= $\{NOTANGLEFLAGS.h\}
NOTANGLE.hxx?=      notangle $\{NOTANGLEFLAGS.hxx\} -R"[[$(notdir $@)]]" \\
  $(filter %.nw,$^) | $\{CPIF\} $@ && noroots $(filter %.nw,$^)
\nwused{\\{NW3JNJ0B-29jjeg-5}}\nwendcode{}\nwbegindocs{34}\nwdocspar

The suffix rules for Haskell is similar to those for C and C++, due to the 
Glasgow Haskell Compiler (GHC) being very close to the C and C++ compilers.
\nwenddocs{}\nwbegincode{35}\sublabel{NW3JNJ0B-3G3VYa-1}\nwmargintag{{\nwtagstyle{}\subpageref{NW3JNJ0B-3G3VYa-1}}}\moddef{defaults for Haskell~{\nwtagstyle{}\subpageref{NW3JNJ0B-3G3VYa-1}}}\endmoddef\nwstartdeflinemarkup\nwusesondefline{\\{NW3JNJ0B-29jjeg-5}}\nwenddeflinemarkup
NOWEB_SUFFIXES+=    .hs
NOTANGLEFLAGS.hs?=  -L
NOTANGLE.hs?=       notangle $\{NOTANGLEFLAGS.hs\} -R"[[$(notdir $@)]]" \\
  $(filter %.nw,$^) | $\{CPIF\} $@ && noroots $(filter %.nw,$^)
\nwused{\\{NW3JNJ0B-29jjeg-5}}\nwendcode{}\nwbegindocs{36}We also note that we do not need any suffix rule for {\Tt{}.lhs\nwendquote} files, for the 
same reason as for the weaving, GHC automatically tangles Haskell's native 
literate files ({\Tt{}.lhs\nwendquote}).

Make also requires slightly modified flags.
We need {\Tt{}-t2\nwendquote} to expand spaces into tabs, because make(1) requires tabs for 
indentation, not spaces.
\nwenddocs{}\nwbegincode{37}\sublabel{NW3JNJ0B-zwVnh-1}\nwmargintag{{\nwtagstyle{}\subpageref{NW3JNJ0B-zwVnh-1}}}\moddef{defaults for Make~{\nwtagstyle{}\subpageref{NW3JNJ0B-zwVnh-1}}}\endmoddef\nwstartdeflinemarkup\nwusesondefline{\\{NW3JNJ0B-29jjeg-5}}\nwenddeflinemarkup
NOWEB_SUFFIXES+=    .mk
NOTANGLEFLAGS.mk?=  -t2
NOTANGLE.mk?=       notangle $\{NOTANGLEFLAGS.mk\} -R"[[$(notdir $@)]]" \\
  $(filter %.nw,$^) > $@ && noroots $(filter %.nw,$^)
\nwused{\\{NW3JNJ0B-29jjeg-5}}\nwendcode{}\nwbegindocs{38}\nwdocspar

For Python, LaTeX, shell scripts and Go, there is no special processing needed, 
we simply use the flags we set in the beginning.
We will now iterate through \emph{all} suffixes, including {\Tt{}.mk\nwendquote} and {\Tt{}.cpp\nwendquote} 
\etc that we have already defined options for.
Since we use the {\Tt{}?=\nwendquote} operator, those we have already defined will be 
ignored.
This has the added benefit that one can simply add {\Tt{}.suf\nwendquote} to 
{\Tt{}NOWEB{\_}SUFFIXES\nwendquote} and this code will automatically generate the default 
settings.
\nwenddocs{}\nwbegincode{39}\sublabel{NW3JNJ0B-34DmtX-1}\nwmargintag{{\nwtagstyle{}\subpageref{NW3JNJ0B-34DmtX-1}}}\moddef{defaults for remaining~{\nwtagstyle{}\subpageref{NW3JNJ0B-34DmtX-1}}}\endmoddef\nwstartdeflinemarkup\nwusesondefline{\\{NW3JNJ0B-29jjeg-5}}\nwprevnextdefs{\relax}{NW3JNJ0B-34DmtX-2}\nwenddeflinemarkup
NOWEB_SUFFIXES+=    .py .sty .cls .sh .go

define default_tangling
NOTANGLEFLAGS$(1)?=
NOTANGLE$(1)?=      notangle $$\{NOTANGLEFLAGS$(1)\} -R"[[$$(notdir $$@)]]" \\
  $$(filter %.nw,$$^) | $$\{CPIF\} $$@ && noroots $$(filter %.nw,$$^)
endef

$(foreach suffix,$\{NOWEB_SUFFIXES\},$(eval $(call default_tangling,$\{suffix\})))
\nwalsodefined{\\{NW3JNJ0B-34DmtX-2}}\nwused{\\{NW3JNJ0B-29jjeg-5}}\nwendcode{}\nwbegindocs{40}\nwdocspar

However, we'd like to add an extra post-processing step for Python:
we'd like to run a Python formatter (such as Black) on the generated source 
code.
\nwenddocs{}\nwbegincode{41}\sublabel{NW3JNJ0B-34DmtX-2}\nwmargintag{{\nwtagstyle{}\subpageref{NW3JNJ0B-34DmtX-2}}}\moddef{defaults for remaining~{\nwtagstyle{}\subpageref{NW3JNJ0B-34DmtX-1}}}\plusendmoddef\nwstartdeflinemarkup\nwusesondefline{\\{NW3JNJ0B-29jjeg-5}}\nwprevnextdefs{NW3JNJ0B-34DmtX-1}{\relax}\nwenddeflinemarkup
NOTANGLE.py+=       && $\{NOWEB_PYCODEFMT\}
\nwused{\\{NW3JNJ0B-29jjeg-5}}\nwendcode{}\nwbegindocs{42}\nwdocspar

We'll add Black as the default Python code formatter.
\nwenddocs{}\nwbegincode{43}\sublabel{NW3JNJ0B-29jjeg-6}\nwmargintag{{\nwtagstyle{}\subpageref{NW3JNJ0B-29jjeg-6}}}\moddef{variables~{\nwtagstyle{}\subpageref{NW3JNJ0B-29jjeg-1}}}\plusendmoddef\nwstartdeflinemarkup\nwusesondefline{\\{NW3JNJ0B-7y8Yg-1}}\nwprevnextdefs{NW3JNJ0B-29jjeg-5}{\relax}\nwenddeflinemarkup
NOWEB_PYCODEFMT?=   black $@
\nwused{\\{NW3JNJ0B-7y8Yg-1}}\nwendcode{}

\nwixlogsorted{c}{{\code{}noweb.mk\edoc{}}{NW3JNJ0B-7y8Yg-1}{\nwixd{NW3JNJ0B-7y8Yg-1}}}%
\nwixlogsorted{c}{{defaults for C and C++}{NW3JNJ0B-1RkDZv-1}{\nwixu{NW3JNJ0B-29jjeg-5}\nwixd{NW3JNJ0B-1RkDZv-1}\nwixd{NW3JNJ0B-1RkDZv-2}}}%
\nwixlogsorted{c}{{defaults for Haskell}{NW3JNJ0B-3G3VYa-1}{\nwixu{NW3JNJ0B-29jjeg-5}\nwixd{NW3JNJ0B-3G3VYa-1}}}%
\nwixlogsorted{c}{{defaults for Make}{NW3JNJ0B-zwVnh-1}{\nwixu{NW3JNJ0B-29jjeg-5}\nwixd{NW3JNJ0B-zwVnh-1}}}%
\nwixlogsorted{c}{{defaults for remaining}{NW3JNJ0B-34DmtX-1}{\nwixu{NW3JNJ0B-29jjeg-5}\nwixd{NW3JNJ0B-34DmtX-1}\nwixd{NW3JNJ0B-34DmtX-2}}}%
\nwixlogsorted{c}{{general tangling rules}{NW3JNJ0B-3Kk0zj-1}{\nwixu{NW3JNJ0B-HHfVd-1}\nwixd{NW3JNJ0B-3Kk0zj-1}}}%
\nwixlogsorted{c}{{rules for C and C++}{NW3JNJ0B-1mCQja-1}{\nwixu{NW3JNJ0B-3iZmQi-1}\nwixd{NW3JNJ0B-1mCQja-1}}}%
\nwixlogsorted{c}{{special rules for different languages}{NW3JNJ0B-3iZmQi-1}{\nwixu{NW3JNJ0B-HHfVd-1}\nwixd{NW3JNJ0B-3iZmQi-1}}}%
\nwixlogsorted{c}{{suffix rule for weaving to PDF}{NW3JNJ0B-4Di5Ld-1}{\nwixu{NW3JNJ0B-dAOzE-1}\nwixd{NW3JNJ0B-4Di5Ld-1}}}%
\nwixlogsorted{c}{{suffix rule for weaving to TeX}{NW3JNJ0B-3EfcsT-1}{\nwixu{NW3JNJ0B-dAOzE-1}\nwixd{NW3JNJ0B-3EfcsT-1}}}%
\nwixlogsorted{c}{{suffix rules for tangling code}{NW3JNJ0B-HHfVd-1}{\nwixu{NW3JNJ0B-7y8Yg-1}\nwixd{NW3JNJ0B-HHfVd-1}}}%
\nwixlogsorted{c}{{suffix rules for weaving documentation}{NW3JNJ0B-dAOzE-1}{\nwixu{NW3JNJ0B-7y8Yg-1}\nwixd{NW3JNJ0B-dAOzE-1}}}%
\nwixlogsorted{c}{{tangle source files with suffix}{NW3JNJ0B-1VVpCX-1}{\nwixu{NW3JNJ0B-3Kk0zj-1}\nwixd{NW3JNJ0B-1VVpCX-1}}}%
\nwixlogsorted{c}{{tangle source files without suffix}{NW3JNJ0B-2tqDpo-1}{\nwixu{NW3JNJ0B-3Kk0zj-1}\nwixd{NW3JNJ0B-2tqDpo-1}}}%
\nwixlogsorted{c}{{variables}{NW3JNJ0B-29jjeg-1}{\nwixu{NW3JNJ0B-7y8Yg-1}\nwixd{NW3JNJ0B-29jjeg-1}\nwixd{NW3JNJ0B-29jjeg-2}\nwixd{NW3JNJ0B-29jjeg-3}\nwixd{NW3JNJ0B-29jjeg-4}\nwixd{NW3JNJ0B-29jjeg-5}\nwixd{NW3JNJ0B-29jjeg-6}}}%
\nwbegindocs{44}\nwdocspar
\nwenddocs{}
