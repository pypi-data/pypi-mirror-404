%- if byCode
Avendo scelto la seguente classe di acciaio e la sensibilità dell'armatura:
\begin{align*}
\text{Classe} &: \text{ \fbox{\VAR{classe}}} & \text{Norma} &: \text{ \underline{\VAR{norma}} } & \text{Sensibilità} &: \text{ \underline{\VAR{sensitivity}} }
\end{align*}
si assumono i seguenti valori carattetistici per il materiale
%- else
Per il materiale si scelgono i seguenti valori caratteristici:
%- endif
\begin{align*}
    f_{sy} &= \VAR{fsy} &   E_{s} &= \VAR{Es} & \gamma_s &= \VAR{gammas} & \sigma^{car}_{sd} &= \VAR{sigmas_max_c}  \\
    e_{su} &= \VAR{esu} &  e_{sy} &= \cfrac{f_{sy}}{E_{s}} = \VAR{esy}
\end{align*}
%- if not byCode
Ai fini del calcolo a fessurazione si assume che l'armatura sia di tipo \text{ \underline{\VAR{sensitivity}} }
%- endif
